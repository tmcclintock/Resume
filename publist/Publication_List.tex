% LaTeX file for resume 
% This file uses the resume document class (res.cls)

\documentclass{res} 
\usepackage{hyperref}
%\usepackage{helvetica} % uses helvetica postscript font (download helvetica.sty)
%\usepackage{newcent}   % uses new century schoolbook postscript font 
\newsectionwidth{0pt}  % So the text is not indented under section headings
\usepackage{fancyhdr}  % use this package to get a 2 line header
\renewcommand{\headrulewidth}{0pt} % suppress line drawn by default by fancyhdr
\setlength{\headheight}{24pt} % allow room for 2-line header
\setlength{\headsep}{12pt}  % space between header and text
\setlength{\headheight}{24pt} % allow room for 2-line header
\pagestyle{fancy}     % set pagestyle for document
\rhead{ {\it T. McClintock}\\{\it p. \thepage} } % put text in header (right side)
\cfoot{}                                     % the foot is empty
\topmargin=-0.5in % start text higher on the page
\usepackage[bottom=1.0in]{geometry}

\begin{document}
\thispagestyle{empty} % this page has no header  
\name{THOMAS MCCLINTOCK -- PUBLICATIONS\\[12pt]}% the \\[12pt] adds a blank line after name
%\address{{\bf Address} \\ 1308 E Spring Street \\
%	University of Arizona \\  Tucson, Arizona, USA  \\ (631) 418-5304}
%\address{{\bf On the web} \\\url{https://github.com/tmcclintock} \\
%  \href{mailto:tmcclintock@email.arizona.edu}{tmcclintock@email.arizona.edu}}

\begin{resume}

{\bf McClintock T.}, et al., 2018, {\it Dark Energy Survey Year 1 Results: Weak Lensing Mass Calibration of redMaPPer Galaxy Clusters}, MNRAS, tmp2591M.

{\bf McClintock T.}, et al., 2018, {\it The Aemulus Project II: Emulating the Halo Mass Function}, submitted to ApJ, arXiv:1804.05866.

Melchior P., Gruen D., {\bf McClintock T.}, et al., 2017, {\it Weak-lensing mass calibration of redMaPPer Clusters in Dark Energy Survey Science Verification Data}, MNRAS, 469, 4899

Simet M., {\bf McClintock T.}, et al., 2017, {\it Weak lensing measurements of the mass--richness relation of SDSS redMaPPer clusters}, MNRAS, 466, 3103.

%{\bf McClintock T.}, Hannah E., Lim K., 2018, {\it Bayesian analysis of frisbee flights}, Eur. J. Phys., in prep.

DeRose J., et al., 2018, {\it The Aemulus Project I: Numerical Simulations for Precision Cosmology}, submitted to ApJ, arXiv:1804.05865.

Zhai, Z., et al., 2018, {\it The Aemulus Project III: Emulation of the Galaxy Correlation Function}, submitted to ApJ, arXiv:1804.05867.

Raghunathan S., et al., 2018, {\it Mass Calibration of Optically Selected DES clusters using a Measurement of CMB-Cluster Lensing with SPTpol Data}, submitted to ApJ, arXiv:1810.10998.

Costanzi M., et al., 2018, {\it Modeling projection effects in optically-selected cluster catalogues}, MNRAS, tmp2549C.

Chang C., et al., 2018, {\it The Splashback Feature around DES Galaxy Clusters: Galaxy Density and Weak Lensing Profiles}, ApJ, 864, 83C.

Friedrich O., et al., 2018, {\it Density split statistics: joint model of counts and lensing in cells}, Phys. Rev. D, 98b, 3508F.

Gruen D., et al., 2018, {\it Density split statistics: Cosmological constraints from counts and lensing in cells in DES Y1 and SDSS}, Phys. Rev. D, 98b, 3507G.

DES Collaboration, 2018, {\it The Dark Energy Survey Data Release 1}, arXiv:1801.03181.

Melia F., {\bf McClintock T.}, 2015, {\it Supermassive black holes in the early universe}, RSPSA, 471, 449.

Melia F., {\bf McClintock T.}, 2015, {\it A test of cosmological models using high-z measurements of H(z)}, AJ, 150, 6.
 
\end{resume} 
\end{document}













