% LaTeX file for resume 
% This file uses the resume document class (res.cls)

\documentclass{res} 
\usepackage{hyperref}
%\usepackage{helvetica} % uses helvetica postscript font (download helvetica.sty)
%\usepackage{newcent}   % uses new century schoolbook postscript font 
\newsectionwidth{0pt}  % So the text is not indented under section headings
\usepackage{fancyhdr}  % use this package to get a 2 line header
\renewcommand{\headrulewidth}{0pt} % suppress line drawn by default by fancyhdr
\setlength{\headheight}{24pt} % allow room for 2-line header
\setlength{\headsep}{12pt}  % space between header and text
\setlength{\headheight}{24pt} % allow room for 2-line header
\pagestyle{fancy}     % set pagestyle for document
\rhead{ {\it T. McClintock}\\{\it p. \thepage} } % put text in header (right side)
\cfoot{}                                     % the foot is empty
\topmargin=-0.5in % start text higher on the page
\usepackage[bottom=1.0in]{geometry}

\begin{document}
\thispagestyle{empty} % this page has no header  
\name{Thomas McClintock\\[0pt]}% the \\[12pt] adds a blank line after name
\address{{\bf Address} \\ 10 Lexington Ct.
  Coram, New York, USA \\ (631) 418-5304}
%	University of Arizona \\  Tucson, Arizona, USA  \\ (631) 418-5304}
\address{{\bf On the web} \\\url{https://tmcclintock.github.io/}\\%https://github.com/tmcclintock} \\
%  \href{mailto:mcclintock@bnl.gov}{mcclintock@bnl.gov}}
  \href{mailto:thmsmcclintock@gmail.com}{thmsmcclintock@gmail.com}}

\begin{resume}
\section{Personal Statement}
\vspace{-8pt}
\hrulefill\\
%\vspace{8pt}
%\vspace{-10pt}
I am a Postdoctoral Research Associate at Brookhaven National Laboratory on Long Island, New York. I work with Dr. An\u{z}e Slosar and Dr. Erin Sheldon on projects within the Dark Energy Survey and LSST Dark Energy Science Collaborations. I am primarily interested in gravitational weak lensing and galaxy cluster cosmology, including mass calibration and the mitigation of limiting systematics. I also construct emulators for cosmology analyses by interpolating between large suites of gravitational $N$-body simulations to create precise physical models.

I completed my PhD in Physics at the University of Arizona working with Professor Eduardo Rozo, and my MSc in High Performance Computing at the University of Edinburgh with Professor David Henty. I received my BA in Physics and Astronomy at Amherst College in Massachusetts while completing a senior thesis with Professor Fulvio Melia.

\section{First Author Publications}
\vspace{-8pt}
\hrulefill\\
{\bf McClintock T.}, et al., 2019, {\it Reconstructing Probability Distributions with Gaussian Processes}, arxiv: 1905.09299\\
{\bf McClintock T.}, et al., 2019, {\it Dark Energy Survey Year 1 Results: Weak Lensing Mass Calibration of redMaPPer Galaxy Clusters}, MNRAS, 482, 1352\\
{\bf McClintock T.}, et al., 2019, {\it The Aemulus Project II: Emulating the Halo Mass Function}, ApJ, 872, 53

%\vspace{-12pt}
\section{Imminent Release}
\vspace{-8pt}
\hrulefill\\
{\bf McClintock T.}, Eifler T., Feng X., in prep. {\it Emulating Weak Lensing Covariance Matrices}\\
{\bf McClintock T.}, et al., in prep., {\it The Aemulus Project IV: Emulating the Halo Bias}\\
{\bf McClintock T.}, et al., in prep., {\it Statistical Analysis of Martian Polar Ice Trough Migration Patterns}\\
{\bf McClintock T.}, Hannah E., Lim K., in prep., {\it Bayesian Analysis of Frisbee Flights}

%\vspace{-12pt}
\section{Significant Contributions}
\vspace{-8pt}
\hrulefill\\
Varga T.~N., DeRose J., Gruen D., {\bf McClintock T.} et al., 2019, {\it Dark Energy Survey Year 1 Results: Validation of Weak Lensing Cluster Member Contamination Estimates from P(z) decomposition}, arxiv:1812.05116\\
DeRose J., et al., 2018, {\it The Aemulus Project I: Numerical Simulations for Precision Cosmology}, arxiv:1804.05865\\
Zhai Z., et al., 2019, {\it The Aemulus Project III: Emulation of the Galaxy Correlation Function}, ApJ, 874, 95\\
Melchior P., Gruen D., {\bf McClintock T.} et al., 2017, {\it Weak-lensing Mass Calibration of redMaPPer Clusters in Dark Energy Survey Science Verification Data}, MNRAS, 469, 4899\\
Simet M., {\bf McClintock T.} et al., 2017, {\it Weak Lensing Measurements of the Mass--Richness Relation of SDSS redMaPPer Clusters}, MNRAS, 466, 3103\\
Melia F., {\bf McClintock T.}, 2015, {\it Supermassive Black Holes in the Early Universe}, RSPSA, 471, 449\\
Melia F., {\bf McClintock T.}, 2015, {\it A Test of Cosmological Models Using High-z Measurements of H(z)}, AJ, 150, 6\\

%\vspace{-12pt}
\section{Contributor}
\vspace{-8pt}
\hrulefill\\
Palmese A., et al., 2019, {\it Stellar Mass as a Galaxy Cluster Mass Proxy: Applications to the Dark Energy Survey redMaPPer Clusters}, arxiv:1903.08813\\
Raghunathan S., et al., 2019, {\it Mass Calibration of Optically Selected DES Clusters Using a Measurement of CMB-Cluster Lensing with SPTpol Data}, ApJ, 872, 170\\
Costanzi M., et al., 2019, {\it Modeling Projection Effects in Optically Selected Cluster Catalogs}, AS, 482, 490\\
Chisari N.~E., et al., 2018, {\it Core Cosmology Library: Precision Cosmological Predictions for LSST}, arxiv:1812.05995\\
Abbot T., et al., 2018, {\it The Dark Energy Survey: Data Release 1}, ApJS, 239, 18\\
Shin T., et al., 2018, {\it Measurements of the Splashback Feature around SZ-selected Galaxy Clusters with DES, SPT, and ACT}, arxiv:1811.06081\\
Chang C., et al., 2018, {\it The Splashback Feature around DES Galaxy Clusters: Galaxy Density and Weak Lensing Profiles}, ApJ, 864, 83\\
Friedrich O., et al., 2018, {\it Density Split Statistics: Joint Model of Counts and Lensing in Cells},  Phys. Rev. D, 98, 3508\\
Gruen D., et al., 2018, {\it Density Split Statistics: Cosmological Constraints from Counts and Lensing in Cells in DES Y1 and SDSS}, Phys. Rev. D, 98, 3507

%\section{\centerline{RECENT WORK}} 
%\vspace{8pt}
%\begin{itemize} \itemsep -2pt % reduce space between items
%  \item {\bf The Aemulus Project}: Developing cosmic emulators for predicting cosmological signals at the sub-percent level. Constructed from a suite of $N$-body simulations using Gaussian Processes.
%  \item {\bf DES Y1 Weak lensing mass calibration}: Determining the mass-richness relation for optically selected galaxy clusters. Performed the mass modeling, developed new software to accurately estimate the covariance of the data, performed statistical analysis on the results using Bayesian statistics and MCMC.
%  \item {\bf LSST-DESC Core Cosmology Library}: Core developer of Core Cosmology Library (CCL) used in projects throughout the Dark Energy Science Collaboration (DESC). CCL is a collection of routines for predicting theoretical quantities related to geometry and large scale structure of the universe. I maintain the primary documentation and help integrate halo models.
%\item {\bf DES SV Weak lensing mass calibration}: Performed the mass modeling and statistical inference of the results. Implemented new code to account for cluster miscentering.
%\item {\bf Bayesian analysis of frisbee flights}: Constrained physical parameters the govern the physics of frisbee flight mechanics, both in simulations and from video data. Co-advised an undergraduate thesis.
%\item {\bf PHYS 105: Introduction to scientific computing}: Designed and taught my own curriculum for the class that is still used today. Integrated Python and data visualization into the course.
%%\item {\bf Supermassive black holes in the early Universe}: Investigated the timescales undertaken for supermassive black holes via Eddington accretion in different cosmological models.
%%\item {\bf Supermassive black holes in the early Universe}: Demonstrated that high redshift supermassive black holes can form via Eddington accretion in the $R_{\rm h} = ct$ Universe, even though this is impossible in $\Lambda$CDM.
%%\item  {\bf A test of cosmological models using high-z measurements of H(z)}: Compared the ability of $\Lambda$CDM and the $R_{\rm h} = ct$ Universe to accomodate the distance-redshift relation given by cosmic chronometer data via model selection techniques.
%%\item  {\bf Modeling the Effectiveness of Tax and Benefits Policies}: A masters thesis in which I parallelized a program that m odels the effect of tax and benefit reforms on individuals' education, employment and savings choices. Completed under Professor David Henty.
%%\item  {\bf The Contribution from Low-Mass X-Ray Binaries to the Positron Annihilation in the Galactic Disk}: A senior thesis that calculated the pair production rate within the accretion disc of LMXBs. Completed under Professor Fulvio Melia. 
%\end{itemize}

\section{Invited Talks, Colloquia \& Seminars}
\vspace{-8pt}
\hrulefill\\
%\vspace{-12pt}
{ Emulation in Cosmological Surveys} - The Aemulus Project\\
{ South American Workshop on Cosmology in the LSST Era} - Galaxy cluster cosmology in DES \& LSST\\
Princeton Astronomy Seminar - Galaxy Cluster Weak Lensing in DES\\
{ Stony Brook Astronomy Seminar} - Galaxy clusters in the Dark Energy Survey\\
{ NYU-CCA short seminar} - Simulating Galaxy Clusters for Cosmology in DES\\
{ Brandeis University Dark Universe Colloquia Series} - Simulations for precision cosmology\\
{ Amherst College Physics Colloquium} - Cosmology with the Dark Energy Survey\\
{ Fermilab Colloquium} - Galaxy clusters in the Dark Energy Survey\\
DES \& LSST-DESC Collaboration Meetings - eight total talks

\section{Committees \& Responsibilities}
\vspace{-8pt}
\hrulefill\\
{\bf DES Cluster Weak Lensing Working Group Coordinator} - organized telecons, inventoried projects within the working group using a wiki, mediated disagreements between members\\
{\bf DES Early Career Scientist Representative} - elected to represent graduate students and post-docs to senior management of the collaboration, organized ECS events and group panels to help members learn about careers in academia and industry\\
{\bf College of Science Student Representative to the GSRP} - elected to represent the College to the graduate student government\\
{\bf UA Physics Grad Council} - elected to represent physics graduate students to department administrators, organized graduate student talk series and pizza lunches with colloquium speakers

\section{Outreach \& Leadership}
\vspace{-8pt}
\hrulefill\\
{\bf Cosmology journal club organizer}- Assembled the weekly reading list and assign readings\\
{\bf Secretary \& Treasurer for the Women in Physics club}- Served as judge for science fairs in local middle and high schools, attended outreach events in local schools and after-school clubs, and worked at the Physics booth at the Tucson Festival of Books\\
{\bf Guest author} - wrote scientific articles for Astrobites, DArchive, and Ultiworld\\
% {DAstrobites guest author for a paper on the cosmic lithium problem in Big Bang nucleosynthesis (\href{https://astrobites.org/2015/07/27/shifting-the-pillars-constraining-lithium-production-in-big-bang-nucleosynthesis/}{article link}).
%  \item College of Science representative to the Graduate and Professional Student Council. Primary achievement was procuring funds to restock department libraries.
%  \item Elected to the Graduate Student Council in the physics department. Coordinated `town hall' meetings between graduate students and department heads, arranged pizza lunches with colloquium speakers, organized weekly graduate student research presentations, and invited colloquium speakers voted on by graduate students.
{\bf Speaker \& co-organizer} -  30th anniversary of the Nobel Prize in Medicine given to Barbara McClintock
        
\section{Awards \& Honors}
\vspace{-8pt}
\hrulefill\\
Galileo Circle Scholar - 2017 \& 2018\\
College of Science Graduate Student Award for Teaching - 2017\\
Outstanding Graduate Student Colloquium Presentation in Spring 2015

%\section{\centerline{EDUCATION}} 
%\vspace{-1pt}
%{\bf University of Arizona}, Department of Physics \hfill Tucson, Arizona, USA\\
%\textit{Doctor of Philosophy}, Physics \hfill Sept. 2012 - Aug. 2018\\
%{\bf University of Edinburgh}, Edinburgh Parallel Computing Centre \hfill Edinburgh, Scotland, UK\\
%\textit{Masters of Science}, High Performance Computing \hfill Sept. 2011 - July 2012\\
%{\bf Amherst College}, Amherst College \hfill Amherst, MA\\
%\textit{Bachelor of Arts cum laude}, Physics and Astronomy \hfill Sept. 2007 - May 2011 \\
%Cumulative GPA: 3.7 \hspace{0.2in}Physics GPA: 3.5 \hspace{0.2in} Astronomy GPA: 3.7
  
%\vspace{0.1in} 
\section{Research Experience} 
\vspace{-8pt}
\hrulefill\\
{\sl University of Arizona Department of Physics} \hfill Tucson, AZ\\
Graduate Student \hfill Sept. 2012 - Present
\begin{itemize} \itemsep -2pt % reduce space between items
  \item Member of The Aemulus Project, which aims to provide emulators for cosmological research. Developed an emulator for the halo mass function accurate at the sub-percent level.
  \item Member of the DES collaboration. Contributed to the cluster calibration, cluster cosmology, trough analysis, and splashback invesitgations.% Advised by Eduardo Rozo. Primary collaborators are Tamas Varga, Peter Melchior, Daniel Gruen, and Erin Sheldon.
  \item Member of LSST-DESC. Presented work on cluster calibration and contributed to the Core Cosmology Library. Leader of the cluster cosmology in Data Challenge 2, which is a project to create a full analysis pipeline on simulated data catalogs.
  \item Analyzed the physics of frisbee flights, both through simulations and video analysis. Used Bayesian inference and MCMC to constrain physical models of flight parameters, and set limits on the resolution and frames-per-second needed to use video data. Co-advised an undergraduate thesis.%with Kevin Lim the thesis of Elizabeth Hannah.
  \item Investigated the $R_{\rm h}=ct$ Universe as a means to accommodate inexplicable physical phenomena including the early growth of supermassive black holes and the redshift evolution of the Hubble constant.% Advised by Fulvio Melia.
\end{itemize}

{\sl University of Edinburgh Edinburgh Parallel Copmuting Centre} \hfill Edinburgh, Scotland, UK\\
Graduate Student \hfill Sept. 2011 - Aug. 2012
\begin{itemize} \itemsep -2pt % reduce space between items
   \item  Developed a parallelized tax and benefits actor simulation to determine optimal tax policies based on individuals' lifestyles.
\end{itemize}

{\sl Amherst College Physics Department} \hfill Amherst, MA\\
Senior Thesis and Research Intern \hfill Sept. 2010 - Aug. 2011
\begin{itemize} \itemsep -2pt
	\item Calculated the production of 511 keV flux from low-mass x-ray binary star systems and determined their contribution to the excess positron annihilation signal observed in the galactic bulge.%Investigated the possibility of low-mass x-ray binary star systems as sources of positrons responsible for the 511 keV flux seen in the galactic bulge.
\end{itemize}

{\sl Brookhaven National Laboratory} \hfill Brookhaven, NY\\
Research Intern \hfill June 2010 - Aug. 2010; June 2009 - Aug. 2009
\begin{itemize} \itemsep -2pt
  \item SULI student scientist at the National Synchrotron Light Source.% under Dario Arena.
  \item Developed control systems for ferromagnetic resonance experiments.
  \item Designed and machined an electromagnet for use in x-ray magnetic circular dichroism experiments. %for the U4B beamline
\end{itemize} 

\section{Teaching Experience} 
\vspace{-8pt}
\hrulefill\\
{\it University of Arizona Department of Physics} \hfill Tucson, AZ
\begin{itemize} \itemsep -2pt
  \item Co-advised an undergraduate thesis researching the physics of frisbee flights.
  \item Developed a new curriculum for the PHYS 105: Introduction to Scientific Computing.
  \item Teaching assistant for PHYS 182: Laboratory Electromagnetism and Optics, PHYS 105: Introduction to Scientific Computing, and PHYS 305: Computational Physics.
  \item Grader for PHYS 321: Theoretical Mechanics.
  \item Tutored college and high school students in mechanics, electromagnetism, optics, and statistical mechanics.
\end{itemize}

{\it Amherst College Department of Physics} \hfill Amherst, MA
\begin{itemize}  \itemsep -2pt
  \item Grader for statistical mechanics, and introduction to electromagnetism.
  \item Resident councilor for three years. Worked two years in first-year dormitories and one year in upperclass housing. Assisted residents adjusting to college life and acted as a liaison between students, faculty, and staff.
  \item Tutored high school AP Physics students.
\end{itemize}

%\section{\centerline{LEADERSHIP AND ACTIVITIES}} 
%\vspace{15pt}
%\begin{itemize}\itemsep -2pt % reduce space between items
%\item Co-organized and spoke at the celebration for the 30th Anniversary of the 
%  Nobel Prize in Medicine given to Barbara McClintock.
%\item Amherst College and University of Arizona Ultimate Frisbee team captain.
%\item Huntington Ultimate Summer League cofounder. Coached high school students 
%  and developed the sport of ultimate frisbee on Long Island.
%\item Clan Gordon Highlanders Marching Band drummer and bagpiper.
%\item Amherst College Music Department stage crew member for musicals and plays.
%\item Amherst Film Club crew member.
 
\end{resume} 
\end{document}













